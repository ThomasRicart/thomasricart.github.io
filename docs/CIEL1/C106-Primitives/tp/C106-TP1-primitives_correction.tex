\documentclass[a4paper,10pt]{report}

\def\packagepath{../../../../../preambule}   % path du package princial
\usepackage{\packagepath/preambule}    % utilisation du fichier de configuration

\def\level{BTS }              % Classe
\def\course{CIEL 1}              % Matière
\def\eval{Primitives -- Correction}

\def\visibleornot{visible}    % visible or invisible
\def\documentpath{./Sources_Latex}
\def\date{C106 - TP1 }
\renewcommand{\arraystretch}{1}  % Ecart dans les tableau

\def\ptsexoA{0}
\def\ptsexoB{0}
\def\ptsexoC{0}
\def\ptsexoD{4}
\def\ptsexoE{1}
\def\ptsexoF{1}
\def\ptsexoG{2}
\def\ptsexoH{1}
\def\ptsexoI{2}
\def\ptsexoJ{2}
\def\ptsexoK{1}

\def\ptstotal{\ptsexoA+\ptsexoB}

\usepackage{hyperref}

\usetikzlibrary{shapes, arrows, positioning}
\usetikzlibrary{shapes.geometric, arrows, positioning, decorations.pathreplacing}

\tikzstyle{etat} = [draw, rounded corners=15pt, minimum width=2.5cm, minimum height=1.2cm, text centered, font=\bfseries]
\tikzstyle{nouveau} = [etat, fill=blue!40]
\tikzstyle{pret} = [etat, fill=green!60]
\tikzstyle{elu} = [etat, fill=yellow!60]
\tikzstyle{bloque} = [etat, fill=red!60]
\tikzstyle{termine} = [etat, fill=white]
\tikzstyle{fleche} = [thick, ->, >=stealth]

\usepackage{float} % Permet d'utiliser [H]
\usepackage[T1]{fontenc}
\begin{document}

%%%%%%%%%%%%%%%%%%%%%%%%%%%%%%%%%%%%%%%%%%%%%%%%%%ù

%\PageGardeSujetBac[Matiere = Numérique et Sciences Informatiques,
%                  Session = 2025,
%                  AffJour=false,
%                  Duree = {3 heures 30},
%                  DernierePage = \pageref{LastPage},
%                  ModeExamen=false
%                  ]
\renewcommand{\labelitemi}{\textbullet} %pour éviter les tirets dans les "itemize" qui apportent confusion avec le signe moins.
%% Début page de garde style BAC

   %%%%%%%%%%%%%%%%%%%%%%%%%%%%%%%%%%%%%%%%%%%%%%%%%%%%%%%%
%\PageGardeSujetBac[clés]

\pagestyle{DS_LP}          % Feuille de style fancy
\NomPrenomNote{}


\exods{\ptsexoA}

\medskip


\begin{enumerate}
    \item Soit la fonction $f_1$ définie par $f_1(x)=3x^2+2x-1$. 
    \begin{enumerate}
        \item Déterminer une primitive de la fonction $f_1$.
        
        \begin{easybox*}{\qquad}{}
        \begin{correction}
        Intégration terme à terme :
        $$F_1(x) = 3 \cdot \frac{x^3}{3} + 2 \cdot \frac{x^2}{2} - x + C = x^3 + x^2 - x + C$$
        
        Une primitive : $\boxed{F_1(x) = x^3 + x^2 - x}$
        \end{correction}
        \end{easybox*}
        
        
        \item Déterminer la primitive de la fonction $f_1$ qui vaut 6 en $x=0$.
        
                \begin{easybox*}{\qquad}{}

        \begin{correction}
        Condition initiale : $F_1(0) = 6$
        $$F_1(0) = 0 + 0 - 0 + C = C = 6$$
        
        Donc : $\boxed{F_1(x) = x^3 + x^2 - x + 6}$
        \end{correction}
            \end{easybox*}
\end{enumerate}
    
    \item $f_2(x)=x^3 - 2x^2 + 5x$
    \begin{enumerate}
        \item Déterminer une primitive de la fonction $f_2$.
        
                \begin{easybox*}{\qquad}{}
\begin{correction}
        Intégration terme à terme :
        $$F_2(x) = \frac{x^4}{4} - \frac{2x^3}{3} + \frac{5x^2}{2} + C$$
        
        Une primitive : $\boxed{F_2(x) = \frac{x^4}{4} - \frac{2x^3}{3} + \frac{5x^2}{2}}$
        \end{correction}
                \end{easybox*}

        \item Déterminer la primitive de la fonction $f_2$ qui vaut 0 en $x=1$.
        
                \begin{easybox*}{\qquad}{}
\begin{correction}
        Condition initiale : $F_2(1) = 0$
        $$F_2(1) = \frac{1}{4} - \frac{2}{3} + \frac{5}{2} + C = 0$$
        
        Calcul du dénominateur commun : $\frac{1}{4} - \frac{2}{3} + \frac{5}{2} = \frac{3-8+30}{12} = \frac{25}{12}$
        
        Donc : $C = -\frac{25}{12}$
        
        Réponse : $\boxed{F_2(x) = \frac{x^4}{4} - \frac{2x^3}{3} + \frac{5x^2}{2} - \frac{25}{12}}$
        \end{correction}
            \end{easybox*}
\end{enumerate} 
\end{enumerate}

\bigskip

\exods{\ptsexoB}

\medskip

Soit $f$ la fonction définie sur $\R$ par $f(x)=(2x+1)\e^{-x}$.

\begin{enumerate}
    \item Etudier le signe de la fonction $f$ sur $\R$
    
            \begin{easybox*}{\qquad}{}
\begin{correction}
    $\e^{-x} > 0$ pour tout $x \in \R$, donc le signe de $f$ dépend de $(2x+1)$.
    
    $2x+1 = 0 \Rightarrow x = -\frac{1}{2}$
    
    
    \end{correction}
            \end{easybox*}

    \item Etudier les variations de la fonction $f$ sur $\R$.
    
            \begin{easybox*}{\qquad}{}
\begin{correction}
    $f(x) = (2x+1)\e^{-x}$
    
    Formule du produit : $f'(x) = 2 \cdot \e^{-x} + (2x+1) \cdot (-\e^{-x}) = \e^{-x}[2 - (2x+1)]$
    
    $f'(x) = \e^{-x}(1-2x)$
    
    $f'(x) = 0 \Rightarrow 1-2x = 0 \Rightarrow x = \frac{1}{2}$
    
    
    
    Maximum en $x = \frac{1}{2}$ : $f(\frac{1}{2}) = 2 \cdot \e^{-1/2} = \frac{2}{\sqrt{e}}$
    \end{correction}
            \end{easybox*}

    \item Déterminer l'équation de la tangente à la courbe représentative de $f$ au point d'abscisse 0.
    
            \begin{easybox*}{\qquad}{}
\begin{correction}
    Point : $f(0) = 1 \cdot \e^0 = 1$ donc $(0, 1)$
    
    Pente : $f'(0) = \e^0(1-0) = 1$
    
    Tangente : $y - 1 = 1(x - 0)$
    
    $\boxed{y = x + 1}$
    \end{correction}
            \end{easybox*}

    \item Montrer que $F(x)=-(2x+3)\e^{-x}$ est une primitive de la fonction $f$.
    
            \begin{easybox*}{\qquad}{}
\begin{correction}
    Dériver $F(x) = -(2x+3)\e^{-x}$ :
    
    $F'(x) = -2 \cdot \e^{-x} - (2x+3) \cdot (-\e^{-x}) = \e^{-x}[-2 + 2x + 3] = (2x+1)\e^{-x} = f(x)$ 
    \end{correction}
            \end{easybox*}

    \item Déterminer la primitive de la fonction $f$ telle que $F(0)=3$.
    
            \begin{easybox*}{\qquad}{}
\begin{correction}
    Forme générale : $F(x) = -(2x+3)\e^{-x} + C$
    
    Condition : $F(0) = 3$
    
    $F(0) = -(3)\e^0 + C = -3 + C = 3 \Rightarrow C = 6$
    
    $\boxed{F(x) = -(2x+3)\e^{-x} + 6}$
    \end{correction}
        \end{easybox*}
\end{enumerate}

\bigskip

\exods{\ptsexoC}

\medskip

\begin{easybox*}{\quad}{}
A faire uniquement si vous 
\begin{itemize}
    \item avez fini les exercices précédents
    \item avez compris parfaitement les exercices précédents
\end{itemize}
\end{easybox*}

\begin{propriete}{Intégration par parties}{}
$$\int u(x)v'(x) dx = u(x)v(x) - \int u'(x)v(x) dx$$

avec
\begin{itemize}
    \item $u$ et $v$ des fonctions dérivables sur un intervalle $I$.
    \item le symbole $\int$ représentera pour le moment le fait de chercher une primitive d'une fonction. 
    \item Ainsi par exemple $\int f(x)dx$ représente une primitive de la fonction $f$.
\end{itemize}



\end{propriete}

\begin{enumerate}
    \item Déterminer une primitive de la fonction $f$ définie par $f(x)=x\e^x$.
    
            \begin{easybox*}{\qquad}{}
\begin{correction}
    Intégration par parties : $u = x$, $dv = \e^x dx$
    
    $du = dx$, $v = \e^x$
    
    $\int x\e^x \, dx = x\e^x - \int \e^x \, dx = x\e^x - \e^x = \e^x(x-1) + C$
    
    Une primitive : $\boxed{F(x) = \e^x(x-1)}$
    \end{correction}
            \end{easybox*}

    \item Déterminer une primitive de la fonction $f$ définie par $f(x)=x\e^{-x}$.
    
            \begin{easybox*}{\qquad}{}
\begin{correction}
    Intégration par parties : $u = x$, $dv = \e^{-x} dx$
    
    $du = dx$, $v = -\e^{-x}$
    
    $\int x\e^{-x} \, dx = -x\e^{-x} - \int (-\e^{-x}) \, dx = -x\e^{-x} + \int \e^{-x} \, dx$
    
    $= -x\e^{-x} - \e^{-x} = -\e^{-x}(x+1) + C$
    
    Une primitive : $\boxed{F(x) = -\e^{-x}(x+1)}$
    \end{correction}
            \end{easybox*}

    \item Déterminer une primitive de la fonction $f$ définie par $f(x)=x\ln(x)$.
    
            \begin{easybox*}{\qquad}{}
\begin{correction}
    Intégration par parties : $u = \ln(x)$, $dv = x \, dx$
    
    $du = \frac{1}{x}dx$, $v = \frac{x^2}{2}$
    
    $\int x\ln(x) \, dx = \frac{x^2}{2}\ln(x) - \int \frac{x^2}{2} \cdot \frac{1}{x} \, dx$
    
    $= \frac{x^2}{2}\ln(x) - \int \frac{x}{2} \, dx = \frac{x^2}{2}\ln(x) - \frac{x^2}{4} + C$
    
    Une primitive : $\boxed{F(x) = \frac{x^2}{2}\ln(x) - \frac{x^2}{4}}$
    \end{correction}
            \end{easybox*}

    \item Déterminer une primitive de la fonction $f$ définie par $f(x)=\ln(x)$.
    
            \begin{easybox*}{\qquad}{}
\begin{correction}
    Réécrire : $\ln(x) = \ln(x) \cdot 1$
    
    Intégration par parties : $u = \ln(x)$, $dv = 1 \, dx$
    
    $du = \frac{1}{x}dx$, $v = x$
    
    $\int \ln(x) \, dx = x\ln(x) - \int x \cdot \frac{1}{x} \, dx = x\ln(x) - \int 1 \, dx$
    
    $= x\ln(x) - x + C$
    
    Une primitive : $\boxed{F(x) = x\ln(x) - x}$
    \end{correction}
        \end{easybox*}

\end{enumerate}



\end{document}